
\chapter{Literature Review}

This literature review analyses the most relevant literature on the development methodology used to build Digital Twins for chemical industries.

The design process of a number of Digital Twin Systems are compared, to show the common elements and challenges in building a digital twin system.

Then, some existing and emerging technologies are reviewed that may be helpful in creating a process for building digital twins.

\section{An Overview of Digital Twins}

Digital Twins are a broad concept that have been used in many disciplines. This article shall use a general definition: A Digital Twin is a set of digital data that mirrors a physical object or system. 

The fundamental characteristics of a digital twin system are that:

\begin{itemize}
    \item A digital twin represents one physical system - i.e the specific physical object must be identifiable~\cite{minerva_digital_2020}.
    \item It stores some ``state" that defines key attributes of the physical system.
    \item It is updated to reflect changes in the physical system.
\end{itemize}

There are a number of characteristics that are strongly associated with digital twin technology. In different contexts, these properties may be assumed to be part of a Digital Twin, and in consequence, different articles may be describing quite different pieces of technology under the same label of ``Digital Twin''.

These include:
\begin{itemize}
    \item Real-time state updates: Digital Twins are often updated in real-time, or near real-time, to reflect changes in the physical system. 
    \item Modelling capabilities: Rather than just storing the state of the system, many Digital Twins also store the constraints and equations that govern the system. This could be implemented with mathematical models or machine learning methods.
    \item Real-time model updates: A specific case of real-time state updates, where the constraints and equations that govern the system are updated in real-time. This borders on the field of self-adaptive sytems % citation needed
    \item Simulation capabilities: The ability to use a model to predict what would happen to the state of the system under different conditions.
    \item Bi-directional communication: Some Digital Twin systems also include control systems, and thus are able to influence the state of the physical system.
    \item Context: The Digital Twin may also store context of the environment the physical system is in.
    \item Agentic ability: The Digital Twin may be desgined to work in a multi-agent system, interacting with other digital twins or acting on behalf of the physical system.
\end{itemize}


For example:

A Digital Twin of a IoT weather station may recieve real-time updates from the weather station. It can then respond to queries for information in the same manner the weather station would, reducing load on the servers and data link of the IoT device. This would be a simple digital twin system, with real-time state updates, a model of the weather station's response to queries, and rudimentary agentic ability.

A Chemical Engineer may create a mathematical first-principles model of a factory, desgined to represent the factory's current state of operations, including degradation of equipment and current feedstock. This could also be considered a rudimentary digital twin, which does not have real-time updates, but accurately models the system and can be used to simulate changes to the system. 

Google Maps can be considered a digital twin of the traffic system. It contains a model of the road network, and uses traffic data as context in real-time to model and predict journey times. It can use this to suggest the best route to take, a form of agentic ability.

A Digital Twin of a mechanical robot may be able to model and simulate the robot's response to different control inputs, and then pass the best control inputs to the robot to execute. This would be a digital twin with real-time state updates, simulation capabilities, and bi-directional communication.


\subsection{Fidelity}

% maybe move to the 3d modelling section


\subsection{Adjacent technologies}

These technologies may not strictly define a digital twin, but they run parallel to some of the goals of digital twins, and are often used together.

\subsubsection{Data Collection \& History}

In most cases, Digital Twin systems are real-time and thus must require data to be collected from the physical system.

% Give some examples to get an idea of the range

Additionally, storing and using historical data can be useful for learning interactions within the system and within the environment, which can be used to improve the Digital Twin model's ability to simulate and predict responses. 

\subsubsection{Mathematical Modelling}

This involves using some kind of mathematical or algebraic formula to represent the interactions within the system, or between the system and it's environment. An example of this is using a set of thermodynamic equations to model the behaviour of a chemical reactor. 

Mathematical models can be used to enrich the data collected from the physical system: in the reactor example, the data collected may be the temperature and pressure of the reactor, but the model can be used to calculate the concentration of reactants and products, which may not be directly measurable.

There are a number of tools for mathematical modelling. One of the industry standards is the Modelica Modelling Language, which has been used to model a variety of systems. However, there are a variety of other algebraic modelling languages, such as Julia's ModelingToolkit.jl, and Python's Pyomo, that can also be used quite generally. There are also specific tools for different domains that use mathematical modelling and provide a more high-level interface.


\subsubsection{Sensor/Data Fusion}

Sensor fusion is the process of combining data from multiple sensors to produce a more accurate representation of the system. This can be used to reduce noise in the data, or to provide more information than any single sensor can provide. Usually, this is done by first modelling the relationships between the sensors, and then using this model to predict the most accurate "true" properties of the system. 

An example of sensor fusion is a virtual reality headset, which uses multiple cameras, accelerometers, and gyroscopes to track the user's head position and orientation. The data from each sensor is combined to produce a more accurate representation of the user's head position and orientation than any single sensor could provide.


\subsubsection{Control}

A digital twin can be used to control the physical system. This is often done by using the digital twin to simulate the effects of different control inputs, and then choosing the best control input to apply to the physical system. This is the domain of Model Predictive Control.

A control system itself can be modelled as, or as part of, a digital twin. % ...so what? It can be it can be used to see how the control ssytem will respond to different conditions? idk


\subsubsection{3D Modelling and Simulation}

3D models can often be used to simulate physical systems, such as newtonian physics, fluid dynamics, heat transfer, or flow. Manipulating a 3D model in a simulation can be a powerful way to create a digital twin. 

Some definitions of Digital Twins focus on 3D models as the required ``physical representation'' of a digital twin. This is a valid and useful way of building digital twins, but digital twins need only the data that is necessary to represent the key characteristics of the physical system. For the weather station d, a 3D model would be unnecessary, but in a digital twin of a robot, bridge, or aeroplane, a 3D model would be very imporant.

\subsubsection{Online Learning}

In the space of Machine Learning, Online Learning refers to the process of updating a machine learning model in real-time as new data comes in, without any need for human input, and usually without a need to retrain the model from scratch. This can be used to encode changes in not just the state but also the behaviour of the physical system, such as changes in the efficiency of a pump, or the degradation of a battery.

Depending on the rate at which data is recieved, retraining a machine learning model from scratch may or may not be feasible. To retrain the model from scratch, historical data must be stored to form the training set. However, techniques from online learning to drift detection and automatic retraining and validation can still be used.

\subsubsection{Self-Adaptive Systems}



\subsubsection{Multi-Agent Systems}











% What characterises digital twin technology

% Use google maps as the digital twin example?

\subsection{Usage in Chemical and Process Engineering}

% Without going into detail, what are they used for?


% Should we talk about success metrics?

\section{Digital Twin Development Methodologies}

\subsection{Case study 1: ...}

\subsection{Case study 2: ...}

\subsection{Case study 3: ...}

\subsection{Discussion}
% Common elements of the design process
% challenges in building a digital twin system
% compatibility, complexity, security, privacy etc
% find some more specific challenges too.



\section{Digital Twin Development Tools in Chemical Engineering}


\subsection{Equation-oriented Modelling Tools}
% comparison



% Potential Comparison Table
% Language
% Supports importing from
% Supports exporting to
% Engineering Packages & Model Libraries
% ODE & PDE Support
% control integration?
% Created
% Popularity (active, # github stars, or something)
% Link to website
% Machine Learning Support
% Value Proposition (point of difference)



% maybe don't have a subsection for each, just a table explaining the differences


\subsubsection{Modelica Modelling Language}

Lots of libraries, https://modelica.org/libraries/, and good integration with other digital twin platforms e.g modelon. Development is spread more thinly and less new stuff coming out.


\subsubsection{Julia ModelingToolkit.jl}

More modern than modelica, seems to have good integration with different solvers


\subsubsection{Python Pyomo}

% SciML, https://github.com/SciML/SBMLToolkit.jl importing
% https://www.youtube.com/watch?v=ZYkojUozeC4


\subsection{Digital Twin Platforms}


% Table for all of these?


\subsubsection{Modelon}
integration with Modelica 

\subsubsection{Collimator}
https://www.collimator.ai/


built on python and Jax

\subsubsection{JuliaSim}
https://info.juliahub.com/products/juliasim


\subsubsection{SimScape}
https://www.mathworks.com/products/simscape.html

\subsubsection{AspenTech} ???

\subsubsection{Honeywell} ???

\subsubsection{Siemens} ???

\subsubsection{AVEVA} ???

\subsubsection{Aspen Hysys} ???

\subsubsection{scilab-XCOS}

% This article is quite relevant https://www.mdpi.com/2227-9717/10/1/21


\section{Emerging Technologies }

% FUnctional Mockup Interface - A way of integrating different models



\subsection{Surrogate Modelling}

\subsection{Operator Networks}



